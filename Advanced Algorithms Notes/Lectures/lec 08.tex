\chapter{Lec 08 - General SSSP Problem}

\section{General SSSP Problem}
In the general formulation of the SSSP problem, the graphs can have edges with negative weights. With negative weights we must be careful about what we mean by shortest path. Let's consider the following graph:

\begin{tikzpicture}[node distance={15mm}, thick, main/.style = {draw, circle}, edge/.style = {->,> = latex'}]
            \node[main] (1) {$1$}; 
            \node[main] (2) [right of=1] {$2$}; 
            \node[main] (3) [below right of=2] {$3$}; 
            \node[main] (4) [above right of=3] {$4$};
            \node[main] (6) [above left of=4] {$6$};

            \draw[edge] (1) -- node[above] {10} (2) ;
            \draw[edge] (2) -- node[above] {-3} (3);
            \draw[edge] (3) -- node[above] {2} (4);
            \draw[edge] (4) -- node[above] {-4} (6);
            \draw[edge] (6) -- node[above] {3} (2);
             
\end{tikzpicture}\newline\newline
This graph contains a \textbf{negative weights cycle}. Let's say that we want to compute the shortest path from $1$ to $2$. If we run Dijkstra algorithm, it will get stucked in the cycle forever. Basically $dist(1, 2) = -\infty$.\newline\newline
Therefore, we need to compute the shortest cycle-free/simple paths. Now the problem is well defined, but it is NP-Hard (problem for which we don't have any polynomial time algorithm).\newline\newline
So let's redefine the SSSP problem in a revised version:
\begin{itemize}
    \item \textbf{input:} A directed weighted graph $G = (V, E)$ and a source vertex $s \in V$.

    \item \textbf{output:} One of the following:
    \begin{enumerate}
        \item $dist(s, v) \,\, \forall v \in V$

        \item A declaration that $G$ contains a negative cycle.
    \end{enumerate}
\end{itemize}
Note that a shortest path \textbf{cannot} contain a cycle, except for 0-weight cycles, but in that case we can just remove all of them.\newline\newline
What needs to be changed in Dijkstra's algorithm in order to make it able to deal with negative-weights edges?\newline\newline
\textbf{Intuition:}\newline\newline
The problem in Dijkstra's algorithm is that it never updates its decisions. $len(v)$ should be an \textbf{estimated distance} and it needs to be updated for every vertex ($n-1$ times because this is the maximum number of edges in a simple path between any two vertices).

\subsection{Bellman-Ford algorithm}
\begin{itemize}
    \item \textbf{input:} A directed graph $G$ with edge weights $w: E \rightarrow \mathbb{R}$, and a source vertex $s \in V$. 
    
    \item \textbf{output:} Either $dist(s, v)$ or a declaration that $G$ contains a negative cycle.

\end{itemize}

\begin{algorithm}
\caption{Bellman-Ford}\label{Bellman-Ford}
    \begin{algorithmic}[1]
    \Procedure{Bellman-Ford ($G, s$)}{}
        \State $len(s) = 0$
        \State $len(v) = \infty \,\, \forall v \neq s$
        \For{$n - 1$ iterations}
            \For{edge $(u, v) \in E$}
                \State $len(v) = min(len(v), len(u) + w(u, v))$
            \EndFor
        \EndFor

        \For{edge $(u, v) \in E$}
            \If{$len(v) > len(u) + w(u, v)$}
                \Return \textit{G contains a negative cycle}
            \EndIf
        \EndFor
    \EndProcedure
    \end{algorithmic}
\end{algorithm}
\textbf{Complexity:} $O(m \cdot n)$.\newline\newline
\textbf{Example:}\newline\newline
\begin{tikzpicture}[node distance={15mm}, thick, main/.style = {draw, circle}, edge/.style = {->,> = latex'}]
            \node[main] (s) {$s$}; 
            \node[main] (a) [right of=s] {$a$}; 
            \node[main] (b) [below right of=a] {$b$}; 
            \node[main] (c) [below of=b] {$c$};
            \node[main] (d) [below left of=c] {$d$};
            \node[main] (e) [left of=d] {$e$};
            \node[main] (f) [above left of=e] {$f$};
            \node[main] (g) [below left of=s] {$g$};

            \draw[edge] (s) -- node[above] {10} (a) ;
            \draw[edge] (s) -- node[above] {8} (g);
            \draw[edge] (g) -- node[left] {1} (f);
            \draw[edge] (f) -- node[above] {$-4$} (a);
            \draw[edge] (f) -- node[above] {$-1$} (e);
            \draw[edge] (e) -- node[above] {$-2$} (b);
            \draw[edge] (b) -- node[left] {$1$} (c);
            \draw[edge] (c) -- node[above] {$3$} (d);
            \draw[edge] (d) -- node[above] {$-1$} (e);
            \draw[edge] (b) -- node[above] {$1$} (a);
            \draw[edge] (a) -- node[above] {$2$} (e);
\end{tikzpicture}
\vspace{5pt}
\begin{center}
    \begin{tabular}{c| c c c c c c c c|c}
        & 0 & 1 & 2 & 3 & 4 & 5 & 6 & 7 & last\\
        \hline\hline
         $s$ & 0 & 0 & 0 & 0 & 0 & 0 & 0 & 0 & 0 \\
         $a$ & $\infty$ & 10 & 10 & 5 & 5 & 5 & 5 & 5 & 5\\
         $b$ & $\infty$ & $\infty$ & $\infty$ & 10 & 6 & 5 & 5 & 5 & 5\\
         $c$ & $\infty$ & $\infty$ & $\infty$ & $\infty$ & 11 & 7 & 6 & 6 & 6\\
         $d$ & $\infty$ & $\infty$ & $\infty$ & $\infty$ & $\infty$ & 14 & 10 & 9 & 9\\
         $e$ & $\infty$ & $\infty$ & 12 & 8 & 7 & 7 & 7 & 7 & 7\\
         $f$ & $\infty$ & $\infty$ & 9 & 9 & 9 & 9 & 9 & 9 & 9\\
         $g$ & $\infty$ & 8 & 8 & 8 & 8 & 8 & 8 & 8 & 8\\
\end{tabular}
\end{center}
\textbf{Correctness of the algorithm:}\newline\newline
Let $len(i, v)$ be the length of a shortest path from $s$ to $v$ that contains at most $i$ edges. Since the shortest path from $s$ to $v$ contains at most $n - 1$ edges, it's sufficient to prove that after $i$ iterations $len(v) \leq len(i, v)$.\newline\newline
We'll prove it by induction on $i$:
\begin{itemize}
    \item \textbf{Base case:} $i = 0$
    \[len(s) = 0 \leq len(0, s) = 0\]
    \[len(v \neq s) = +\infty = len(0, v \neq s)\]

    \item \textbf{Inductive hypothesis:} $len(v) \leq len(k, v) \,\, \forall \, 1 \leq k \leq i$
\end{itemize}
Take $i \geq 1$ and take a shortest path from $s$ to $v$ with at most $i$ edges. Let $(u, v)$ be the last edge of this path. Then:
\[len(i, v) = len(i - 1, u) + w(u, v)\]
By inductive hypothesis, $len(u) \leq len(i - 1, u)$. In the $i$-th iteration we update $len(v)$ as follows:
\[len(v) = min(len(v), len(u) + w(u, v))\]
\[len(u) + w(u, v) \leq len(i - 1, u) + w(u, v) = len(i, v)\]
Therefore, $len(v) \leq len(i, v)$.

