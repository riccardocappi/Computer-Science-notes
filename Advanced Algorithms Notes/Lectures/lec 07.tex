\chapter{Lec 07 - Shortest Path}

\section{Definitions and Terminology}
\begin{itemize}
    \item Given a weighted graph, the \textbf{length} of a path $p = v_1, v_2, ..., v_k$ is defined as $len(P)=\sum_{i=1}^{k-1}w(v_i, v_{i+1})$

    \item A \textbf{shortest path} from a vertex $u$ to vertex $v$ is a path with minimum length among all $u-v$ paths.

    \item The \textbf{distance} between 2 vertices $s$ and $t$, denoted as $dist(s, t)$ is the length of a shortest path from $s$ to $t$. If there is no path at all from $s$ to $t$, then $dist(s, t) = \infty$
\end{itemize}

\section{Problem: Shortest Path}
Given a directed, weighted graph, a source vertex $s \in V$ and a destination $t \in V$, compute the shortest path from $s$ to $t$\footnote{In directed graph, in general, $dist(u, v) \neq dist(v, u)$}.\newline\newline
\textbf{Applications:}
\begin{itemize}
    \item Road networks (Google maps)
    \item Routing in computer networks (e.g. internet)
    \item ...
\end{itemize}

\section{Single-Source Shortest Paths (SSSP)}
Description of the problem:
\begin{itemize}
    \item \textbf{input:} A directed, weighted graph $G$ with edge weights $w: E \rightarrow \mathbb{R}$ and a source vertex $s \in V$

    \item \textbf{output:} $dist(s, v) \,\, \forall v \in V$
\end{itemize}
Observations:
\begin{itemize}
    \item No algorithms are known that run asymptotically faster than the best SSSP algorithm in the worst case.

    \item We'll work with \textbf{directed} graphs, but all the algorithms that we'll see can be adapted easily for undirected graphs 
\end{itemize}
\subsection{Special case: non-negative edge weights}
We've already solved a special case of this special case, that is, when all the weights are $w = 1$ it can be solved in linear time using BFS.\newline\newline
Can we modify BFS in order to deal also with weights greater that 1 ? A first simple solution can be to replace and edge with weight $w = n$ with $n$ edges of weight $w = 1$. However, with this implementation, the size of the transformed graph can be much bigger than the size of the starting graph and the complexity can grow exponentially.\newline\newline
\textbf{Intuition of a new algorithm:}\newline\newline
The arc $(s, v)$, at the first step, must be the shortest path from $s$ to $v$ since the first segment of \textbf{any other} path is already longer. A similar reasoning works in the next steps.\newline\newline
\textbf{Dijkstra algorithm:}
\begin{itemize}
    \item \textbf{input:} directed $G$, $s \in V$, $w: E \rightarrow R_{>0}$

    \item \textbf{output:} $len(v) = dist(s, v) \,\, \forall v \in V$
\end{itemize}
\begin{algorithm}
\caption{Dijkstra}\label{Dijkstra}
    \begin{algorithmic}[1]
    \Procedure{Dijkstra ($G, s$)}{}
        \State $\text{X} = \{s\}$
        \State $len(s) = 0$
        \State $len(v) = \infty$
        \While{There is an edge $(v, w)$ with $v \in X$ and $w \notin X$}
            \State $(v^{*}, w^{*}) = $ such an edge minimizing $len(v) + w(v, w)$

            \State add $w^{*}$ to X

            \State $len(w^{*}) = len(v^{*}) + w(v^{*}, w^{*})$
        \EndWhile
    \EndProcedure
    \end{algorithmic}
\end{algorithm}
At each iteration, $X$ contains the node whose shortest paths are known.\newline\newline
Dijkstra does \textbf{not} work on graphs with negative edge weights.\newline\newline
\textbf{Correctness:}
The proof is performed by induction on $|X|$.
\begin{itemize}
    \item \textbf{Invariant:} $\forall x \in X$, $len(x) = dist(s, x)$

    \item \textbf{Base case:} $|X| = 1$ it's trivial

    \item \textbf{Inductive hypothesis:} Invariant is true $\forall |X| = k \geq 1$
\end{itemize}
Let $v$ be the \textbf{next} vertex added to $X$ and $(u, v)$ the arc. The shortest path from $s$ to $u$ $+ (u, v)$ is a path from $s$ to $v$ of length $\pi(v) = min_{(u, v): u \in X, \, v \notin X}len(u) + w(u, v)$. Let's consider any path $P$ from $s$ to $v$, we'll show that $P$ is not shorter than $\pi(v)$. Let $(x, y)$ be the first arc in $P$ that traverses $X$ and let $P^{'}$ be the sub-path from $s$ to $x$. Since all the weights are non-negative, it follows that:
\[len(P) \geq len(P^{'}) + w(x, y)\]
By Inductive hypothesis we know that $len(x): \,\, x \in X = dist(s, x)$:
\[len(P^{'}) + w(x, y) \geq len(x) + w(x, y)\]
By definition of $\pi$ we obtain:
\[len(x) + w(x, y) \geq \pi(y)\]
But since the algorithm always selects the minimum edge:
\[\pi(y) \geq \pi(v)\]
\textbf{Complexity:} $O(m \cdot n)$. This is an inefficient implementation since it keeps searching the minimum edge checking all the edges $n$ times. It can be speeded up using heaps:

\begin{algorithm}
\caption{Dijkstra-Heaps}\label{Dijkstra-Heaps}
    \begin{algorithmic}[1]
    \Procedure{Dijkstra ($G, s$)}{}
        \State $X = \emptyset$
        \State $H = \text{empty heap}$
        \State $key(s) = 0$
        \For{$v \neq s$}
            \State $key(v) = \infty$
        \EndFor
        \For{$v \in V$}
            \State insert $v$ into $H$
        \EndFor

        \While{$H$ is not empty}
            \State $w^* = \text{extractMin}(H)$
            \State add $w^*$ to $X$
            \State $len(w^*) = key(w^*)$
            \For{each edge $(w^*, y)$ such that $y \notin X$}
                \State $key(y) = min(key(y), len(w^*) + w(w^*, y)) \quad //\text{Update the heap}$
            \EndFor
        \EndWhile
    \EndProcedure
    \end{algorithmic}
\end{algorithm}
The complexity of this new implementation is $O((n + m) log\,n)$. There are $O(n + m)$ operations on the heap.

