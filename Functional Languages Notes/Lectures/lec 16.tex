\chapter{Lec 16 - Evaluation rule}

\section{Addition term}
Before talking about the evaluation rules, let's add the $+$ operator to our expressions grammar:
\begin{grammar}
    <e> ::= <x>
    \alt <λx.e> 
    \alt <e e>
    \alt <Let x = e in e>
    \alt <e + e>
\end{grammar}
The type-checking rule for the addition (only between integers) is the following:
\[
    \infer{\Gamma \vdash e_{1} + e_{2}:int}{
    \begin{gathered}
        \Gamma \vdash e_{1}: int \\
        \Gamma \vdash e_{2}:int
    \end{gathered}}
\]

\section{Evaluation rules}
In the C language when a program is compiled the compiler checks syntax and types and produces an executable. Then, when we \textbf{run} the program, the executable is \textbf{evaluated}. Conversely, in python the program is just evaluated, without compiling it.\newline\newline
Let's talk about evaluation of our simple form of expressions.\newline\newline
In the world of evaluation the scope is defined as follows:
\begin{grammar}
    <Δ> ::= Ø
    \alt Δ,(x ⟶ v)
\end{grammar}
The scope $\Delta$ can either be:
\begin{itemize}
    \item empty
    \item extended with a binding for a variable $x$ which computes a \textbf{value} $v$ 
\end{itemize}
$\Gamma$ binds variables to types, while $\Delta$ binds variables to values.\newline\newline
The evaluation rule for the $+$ term previously defined is the following:
\[ \infer{\Delta \vdash e_{1} + e_{2} \rightarrow v}{
\begin{gathered}
    \Delta \vdash e_{1} \rightarrow v_{1}\\
    \Delta \vdash e_{2} \rightarrow v_{2} \quad v = v_{1} + v_{2} 
\end{gathered}}\]
In a scope $\Delta$, the expression $e_{1} + e_{2}$ computes v if, in the same scope $\Delta$, $e_{1}$ computes $v_{1}$, $e_{2}$ computes $v_{2}$ and $v = v_{1} + v_{2}$ where the $+$ operator is the one defined among integers. So, we first evaluate $e_{1}$, then $e_{2}$ and the result is the sum of the two. Note that $v_{1}$ and $v_{2}$ must be numbers.
\subsection{Evaluation of lambdas}
If we write a lambda in the F\# interactive window it will be evaluated as follows:
\begin{lstlisting}[style=FSharpStyle]
    fun x -> x + 1;;
    val it : x:int -> int
\end{lstlisting}
The lambda can't be evaluated until it is called by passing an argument for the parameter $x$. So, in order to define an evaluation rule for lambdas, we need some kind of memory representation that stores the lambda \textbf{blocked} with the scope where it is defined (lexical scoping). This structure is called \textbf{closure}. Then, the evaluation rule is:
\[
\infer{\Delta \vdash \lambda x.e \rightarrow <\lambda x.e;\Delta>}{\emptyset}\]
where $<\lambda x.e;\Delta>$ is a pair composed by the lambda and the original scope where the lambda is defined (closure).
\subsection{Values definition}
A value $v$ is defined by the following BNF grammar:
\begin{grammar}
    <v> ::= lit
    \alt < λx.e ; Δ >
\end{grammar}
$v$ can either be:
\begin{itemize}
    \item a literal (4, true, false, 8.9, ...)
    \item a closure: it consists of a lambda expression and the scope where it is defined.
\end{itemize}

\subsection{Evaluation rules of the other expression terms}
\begin{itemize}
    \item \textbf{Variables identifiers:}
    \[
    \infer{\Delta \vdash x \rightarrow v}{
    \begin{gathered}
        x \in domain(\Delta)\\
        \Delta(x) = v
    \end{gathered}}
    \]

    \item \textbf{Let binding:}
    \[
    \infer{\Delta \vdash let \, x = e_{1} \, in \, e_{2} \rightarrow v_{2}}{\begin{gathered}
        \Delta \vdash e_{1} \rightarrow v_{1}\\
        \Delta, (x\rightarrow v_{1}) \vdash e_{2} \rightarrow v_{2}
    \end{gathered}}
    \]
    Let's consider the following F\# code:
    \begin{lstlisting}[style=FSharpStyle]
        let a = 4 in a
        //val it : int = 4
    \end{lstlisting}
    In this case $e_{1} = 4$ and it is evaluated to 4. Then, the variable $a$ is associated to the evaluated value ('4') and it is added to the scope $\Delta, (a \rightarrow 4)$. Finally, $e_{2}$ is evaluated: since $e_{2} = a$ and $a \rightarrow 4$, the whole expression produces 4.

    \item \textbf{Application:}
    \[
    \infer{\Delta \vdash e_{1} \, e_{2} \rightarrow v_{0}}{\begin{gathered}
        \Delta \vdash e_{1} \rightarrow <\lambda x.e_{0};\Delta_{0}>\\
        \Delta \vdash e_{2} \rightarrow v_{2}\\
        \Delta_{0}, (x \rightarrow v_{2}) \vdash e_{0} \rightarrow v_{0} 
    \end{gathered}}
    \]
    \begin{itemize}
        \item the left part ($e_{1}$) of an application must produce a closure
        \item the right part ($e_{2}$) can produce anything
    \end{itemize}
    This is $\beta$-reduction, that is, how applications works at run-time in functional languages.\newline\newline
    The whole application produces the evaluation of the body $e_{0}$ of the lambda defined in the original environment $\Delta_{0}$, adding a binding for the parameter $x$ to the value that $e_{2}$ computes.
    \begin{lstlisting}[style=FSharpStyle]
        let k = 1

        let f = fun x -> x + k
                
        let n = f 2
        //val n : int = 3
    \end{lstlisting}
    When we call $f$:
    \begin{itemize}
        \item The evaluation of $f$ produces a closure with the scope having $k\rightarrow 1$
        \item $e_{2}=2 \rightarrow 2$
        \item $x$ is bound to 2 ($\Delta,(x\rightarrow 2)$). So, the application produces 3
    \end{itemize}
\end{itemize}
