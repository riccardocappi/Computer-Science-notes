\chapter{Lec 22 - Analysis of Randomized Quicksort}

\section{Randomized Quicksort}
\begin{algorithm}
\caption{Randomized Quicksort}\label{RQS}
    \begin{algorithmic}[1]
    \Procedure{RandQuicksort($S$)}{}
        \If{$|S| \leq 1$}
            \Return $S$
        \EndIf
        \State $p = \text{random($S$)}\quad \text{// pick a pivot element uniformly at random from $S$}$
        \State $S_1 = \{x \in S\,\, \text{s.t. }x < p\}$
        \State $S_2 = \{x \in S\,\, \text{s.t. } x > p\}$
        \State $z_1 = RandQuicksort(S_1)$
        \State $z_2 = RandQuicksort(S_2)$
        \State \Return $z_1, z_2$
    \EndProcedure   
    \end{algorithmic}
\end{algorithm}
It is a \textit{LAS VEGAS} algorithm (we choose the pivot at random).\newline\newline
Suppose that $p$ is always the median of $S$, then:
\[
    T_{RQS}(n) = 
    \begin{cases}
        2T_{RQS}(\frac{n}{2}) + O(n) & n > 1 \\
        0 & n\leq 1
    \end{cases}
\]
where $n = |S|$. Then $T_{RQS}(n) = O(nlog\,n)$.\newline\newline
However, $p$ is the median with probability $\frac{1}{n}$, very low. Actually, there exists a deterministic linear algorithm to compute the median. This implies that this version of deterministic Quicksort has complexity $O(nlog\,n)$. The problem is that the hidden constant is very high, thus it is inefficient in practice.\newline\newline
Fortunately, we don't really need that the random choice hits always the median. For example, let's see what happens if the pivot is always chosen between the $(\frac{n}{4} + 1)$-th order statistics and the $(\frac{3}{4}n)$-th order statistics. By looking at the recursion tree, if such a pivot is always chosen, all the root-leaf paths are no longer than $log_{\frac{4}{3}}n$. This is because $|S|$ after $i$ \textit{successes} is $\leq (\frac{3}{4})^in$. This implies that $T_{RQS} = O(nlog\,n)$. It is not necessary that $S_1$ and $S_2$ are perfectly balanced.\newline\newline
If we prove that the depth of the recursion tree is $O(log\,n)$ w.h.p. then $T_{RQS} = O(nlog\,n)$ w.h.p.

\section{Analysis}
Let's call the event $E = \textit{lucky choice of the pivot}$, that is, pivot chosen between the $(\frac{n}{4} + 1)$-th order statistics and the $(\frac{3}{4}n)$-th order statistics. Then:
\[P(E) = \left(\frac{\frac{3}{4} - (\frac{n}{4} + 1) + 1}{n}\right) = \frac{1}{2}\]
Fix \textbf{one} root-leaf path $P_1$:\newline\newline
\textbf{Lemma:} $P(|P_1| > a \cdot log_{\frac{4}{3}}n) < \frac{1}{n^3}$.\newline\newline
If this lemma is true, it follows that:
\begin{enumerate}
    \item Given the event $E_i = \textit{the path $p_i$ has length } > a \cdot log_{\frac{4}{3}}n$:
    \[P(\exists \textit{ a path with length } > a \cdot log_{\frac{4}{3}}n) = P\left(\bigcup_{i=1}^n E_i\right) \leq \sum_{i=1}^n P(E_i) < \frac{1}{n^2}\]

    \item Therefore, the probability that all the root-leaf paths have length $\leq a \cdot log_{\frac{4}{3}}n$ is the following:
    \[P(\textit{all the root-leaf paths have length $\leq a \cdot log_{\frac{4}{3}}n$}) \geq 1 - P(\exists \textit{ a path with length } > a \cdot log_{\frac{4}{3}}n) \geq 1 - \frac{1}{n^2}\]
\end{enumerate}
This would imply that $T_{RQS}(n) = O(nlog\,n)$ w.h.p.\newline\newline
It remains to prove the Lemma above.\newline\newline
\textbf{Proof:} Given a path $\Pi$, let $l = a \cdot log_{\frac{4}{3}}n$. We want to study the event $E =$ \textit{in the first l nodes of $\Pi$ there have been $< log_{\frac{4}{3}}n$ lucky choices}.
\begin{itemize}
    \item $X_i \quad 1 \leq i \leq l$
    \item $X_i = 1$ if at the $i$-th node of $\Pi$ there is a \textit{lucky choice} of the pivot. 
    \item $P(X_i = 1) = \frac{1}{2} \quad \forall i$.
    \item $X_i$ are independent.
\end{itemize}
We want the probability $P(\sum_{i=1}^l X_i) < log_{\frac{4}{3}}n$ to be bound.\newline\newline
Given $X = \sum_{i=1}^l X_i$, its expected value is defined as follows:
\[\mu = E[X] = E\left[\sum_{i=1}^l X_i\right] = \sum_{i=1}^l E[X_i] = \sum_{i=1}^l \frac{1}{2} = \frac{a}{2}log_{\frac{4}{3}}n\]
Now, let's apply the following Chernoff bound:
\[P(X < (1 - d)\mu) < e^{-\mu\frac{d^2}{2}} \quad 0 < d \leq 1\]
We want $(1 - d)\mu = log_{\frac{4}{3}}n$:
\[(1-d)\frac{a}{2}log_{\frac{4}{3}}n = log_{\frac{4}{3}}n\]
One possible solution is:
\begin{itemize}
    \item $a = 8$
    \item $d = \frac{3}{4}$
\end{itemize}
\[
    \begin{split}
        P(X < log_{\frac{4}{3}}n) & < e^{-\frac{8}{4}log_{\frac{4}{3}}n \cdot \frac{9}{16}} \\
        & = e^{-log_{\frac{4}{3}}n\cdot\frac{9}{8}}\\
        & < e^{-log_{\frac{4}{3}}n}\\
        & = e^{-\frac{ln\,n}{ln\,\frac{4}{3}}}\\
        & = \left(e^{-ln\,n}\right)^{\frac{1}{ln\,\frac{4}{3}}}\\
        & = \left(\frac{1}{n}\right)^{1/ln\,\frac{4}{3}}\\
        & < \frac{1}{n^3}
    \end{split}
\]
